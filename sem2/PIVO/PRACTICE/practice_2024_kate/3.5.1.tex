

%Пример
\begin{SCn}
\begin{small}

\scnheader{Предметная область логических моделей решения задач}
\scnsuperset{дочерняя предметная область*:}

\begin{scnsubdividing}
    \scnitem{Предметная область логических языков }
    \scnitem{Предметная область логического вывода }
\end{scnsubdividing}

\begin{scnsubdividing}
    \scnitem{классический дедуктивный вывод}
        \scntext{пояснение}{Классический дедуктивный вывод всегда дает достоверный результат. Дедуктивный вывод включает в себя прямой и обратный и логический вывод, все виды силлогизмов  и так далее.}
    \scnitem{индуктивный вывод}
        \scntext{пояснение}{Индуктивный вывод предоставляет возможность в процессе решения использовать различные предположения, что делает его удобным для использования в слабоформализованных и трудноформализуемых предметных областях.}
    \scnitem{абдуктивный вывод}
        \scntext{пояснение}{Под абдуктивным выводом в искусственном интеллекте понимается вывод объяснения некоторого события, ставшего неожиданным для системы.}
    \scnitem{нечеткая логика}
        \scntext{пояснение}{Теория нечетких множеств и нечетких логик, также применяется в системах, связанных с трудноформализуемыми предметными областями. Здесь импликативные высказывания могут рассматриваться как "если истинна посылка", то с некоторой вероятностью (часто или редко) истинно заключение.}
    \scnitem{логика умолчаний}
        \scntext{пояснение}{Логика умолчаний применяется для того, чтобы оптимизировать процесс рассуждений, дополняя процесс достоверного вывода вероятностными предположениями, когда вероятность ошибки крайне мала.}
    \scnitem{темпоральная логика}
        \scntext{пояснение}{Применение темпоральной логики является актуальным для нестатичных предметных областей, в которых истинность того или иного утверждения меняется со временем, что существенно влияет на ход решения какой-либо задачи.}
\end{scnsubdividing}

\scnheader{Язык SCL}
\scnidtf{Подъязык SC-кода для записи логических утверждений}  
\scnheader{Абстрактная scl-машина}
\scnidtf{Машина логического вывода и относится к классу абстрактных sc-машин}
\scnheader{Абстрактная scl-машина}
    
\begin{scnrelfromset}{декомпозиция абстрактного sc-агента}
    \scnitem{Абстрактный sc-агент применения правила вывода}
        \scntext{пояснение}{Задачей абстрактного sc-агента применения правила вывода является применение заданного правила вывода с заданными логическими формулами.}
        \scnrelfrom{изображение}{\scnfileimage[25em]{images/121.png}}
    \scntext{пояснение}{SCg-текст. Формализация правила вывода Modus ponens.}


    \scnitem {SCg-текст. Формализация правила вывода Modus ponens}

    \scnitem{Абстрактный sc-агент эквивалентных преобразований логической формулы}
        \scntext{пояснение}{Задачей абстрактного sc-агента эквивалентных преобразований логической формулы является применение некоторых правил, которые приводят логическую формулу в определенный вид.}
    \scnitem{Абстрактный sc-агент прямого логического вывода}
        \scntext{пояснение}{Задачей абстрактного sc-агента прямого логического вывода является генерации новых знаний на основе некоторых логических утверждений.}
\scnrelfrom{изображение}{\scnfileimage[25em]{images/122.png}}
          \scntext{пояснение}{SCg-текст. Спецификация агента прямого логического вывода.}
\scnitem{Абстрактный sc-агент обратного логического вывода}
        \scntext{пояснение}{Задачей абстрактного sc-агента обратного логического вывода является проверка гипотез.}
          \scntext{URL}{https://github.com/ostis-ai/scl-machine}

\end{scnrelfromset}

\scnheader{Абстрактный sc-агент эквивалентных преобразований логической формулы }
\begin{scnrelfromset}{декомпозиция абстрактного sc-агента}
    \scnitem{Абстрактный sc-агент преобразования формулы в конъюнктивную нормальную форму}
    \scnitem {Абстрактный sc-агент преобразования формулы в дизъюнктивную нормальную форму}
    \scnitem{Абстрактный sc-агент применения законов Де Моргана}
    \scnitem{Абстрактный sc-агент эквивалентных преобразований логической формулы по определению}
    \scnitem{Абстрактный sc-агент применения свойств отрицания логических формул}
    \scnitem{Абстрактный sc-агент применения закона идемпотентности логических формул}
    \scnitem{Абстрактный sc-агент применения закона коммутативности логических формул}
    \scnitem{Абстрактный sc-агент применения закона ассоциативности логических формул}
    \scnitem{Абстрактный sc-агент применения закона поглощения логических формул}
    \scnitem{Абстрактный sc-агент применения закона противоречия логических формул}
    \scnitem{Абстрактный sc-агент применения закона двойного отрицания логических формул}
    \scnitem{Абстрактный sc-агент применения закона расщепления логических формул}
\end{scnrelfromset}

\scnheader{Логический язык}
\scnidtf{Формальный язык, предназначенный для воспроизведения логических форм контекстов естественного языка, а также выражения логических законов и способов правильных рассуждений в логических теориях, строящихся в данном языке}
\scnheader{Пролог}
\scnidtf{Язык и система логического программирования}
\scntext{пояснение}{База знаний системы Пролог содержит информацию в виде предикатов. В логическом программировании, реализованном в Прологе, используется только одно правило вывода — правило резолюции.}
 
\scnheader{Предметная область логических формул, высказываний и формальных теорий}
\scnsuperset{дочерняя предметная область*:}

\begin{scnsubdividing}
    \scnitem{Предметная область логических языков }
    \scnitem{Предметная область логического вывода }
\end{scnsubdividing}

\end{small}
\end{SCn}